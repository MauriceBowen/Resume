%%%%%%%%%%%%%%%%%%%%%%%%%%%%%%%%%%%%%%%
% Deedy CV/Resume
% XeLaTeX Template
% Version 1.0 (5/5/2014)
%
% This template has been downloaded from:
% http://www.LaTeXTemplates.com
%
% Original author:
% Debarghya Das (http://www.debarghyadas.com)
% With extensive modifications by:
% Vel (vel@latextemplates.com)
%
% License:
% CC BY-NC-SA 3.0 (http://creativecommons.org/licenses/by-nc-sa/3.0/)
%
% Important notes:
% This template needs to be compiled with XeLaTeX.
%
%%%%%%%%%%%%%%%%%%%%%%%%%%%%%%%%%%%%%%
\documentclass[letterpaper]{clinton-resume}

\begin{document}
%---------HEADER---------
\lastupdated
\namesection{Clinton}{Bowen}{ % Your name
\urlstyle{same}\url{https://www.torrho.com} \\ % Your website, LinkedIn profile or other web address
\href{mailto:clinton.bowen@gmail.com}{Clinton.Bowen@gmail.com} | 818.687.1941 % Your contact information
}
%---------HEADER---------
%---------LEFT COLUMN,PAGE 1---------
\begin{minipage}[t]{0.33\textwidth}
\section{Education} 

\subsection{Cal State Northridge}

\descript{Masters in Applied Mathematics}
\location{Expected May 2015}

\sectionspace % Some whitespace after the section

\descript{BS in Applied Mathematics}
\location{May 2010}

\sectionspace

\section{Links} 
%\faGithubSign: \href{https://github.com/MauriceBowen}{\bf GitHub} \\
\faLinkedinSign: \href{https://www.linkedin.com/pub/clinton-bowen/20/161/91?trk=pub-pbmap}{\bf LinkedIn} \\
%\faTwitterSign: \href{https://twitter.com/mauricebeauxen}{\bf @mauricebeauxen\_das} \\

\section{Software}
\subsection{Languages}
%\begin{tighttabular}
C\\
C++\\
C\#\\
Java\\
\LaTeX\\
Mathematica\\
Matlab\\
Python\\
SQL (MySQL, PostgreSQL, SQLite)\\
%PHP\\
R\\
%\end{tighttabular}

\sectionspace % Some whitespace after the section

\subsection{Frameworks \& Libraries}
%\begin{tighttabular}
.NET\\
Sage Python\\
SciPy\\
RSA BSAFE\\
Gurobi\\
CPLEX\\
Django (and GeoDjango)\\
PeachFuzz\\
MiniFuzz\\
%\end{tighttabular}
\sectionspace % Some whitespace after the section
\subsection{Development Operations}
IC-Agile Certified Professional\\
Secure Development Lifecycle Practitioner\\
Top Secret SCI Active (No FSP)\\
\sectionspace % Some whitespace after the section
\section{Skills}
\subsection{System Engineering}
\location{GPS}
Subject Matter Expert (SME) on C/A and CNAV
\subsection{Cryptography + Cyber-Sec}
\location{Engineering Standards}
SME on FIPS 140 to 202\textbullet{} Cyber-Security Framework \textbullet{} SME on Special Publications 800 Series \textbullet{} RFCs \textbullet{} CNSSPs \textbullet{} DODAF\\
\end{minipage}
%---------LEFT COLUMN,PAGE 1---------
%---------RIGHT COLUMN,PAGE 1---------
\begin{minipage}[t]{0.66\textwidth}
\section{Experience}

\runsubsection{Booz Allen Hamilton Engineering Services, LLC}
\descript{| Technologist}

\location{June 2010 – Present | El Segundo, CA}
\vspace{\topsep} % Hacky fix for awkward extra vertical space
\begin{tightitemize}
\item Directing weekly technical meetings between a team of software developers and clients for project management
	\begin{itemize}
	\item Capture customer input into software development and system engineering requirements and tasks
	\item Provide schedule and progress of software development \& system engineering tasks and backlog items
	\item Illustrate and present system designs and constraints to customers in DOD Architectural Framework formats
	\item Prioritize software development tasks for software development team
	\end{itemize}
\item Provides mentorship for software development interns
	\begin{itemize}
	\item Issue tasks for interns
	\item Provide guidance for completion of tasks
	\end{itemize}
\item Provided cryptographic analysis for a GPS CNAV project
	\begin{itemize}
	\item Identified feasible cryptographic solutions
	\item Assisted in drafting a cryptographic protocol for authentication of associated data and high level overview of key management
	\item Consulted and developed software for prototyping the cryptographic concept
	\end{itemize}
\item Designed a SOAP software architecture for GPS SAASM Mission Planning System
\item Designed C\# framework, ATLAS, for internal use within the Booz Allen Hamilton Advanced Research and Development office
\item Desgined, developed, \& tested a MATLAB Reed-Solomon error correction code library without the MATLAB Communication Toolbox
	\begin{itemize}
	\item Allows for arbitrary $(n,k)$ code encoded using Galois fields
	\item Uses a Berlekamp-Welch decoding scheme
	\end{itemize}
\item Developed \& demonstrated a cryptographic use case using SHA-3 based algorithms in embedded C software for a PIC24HJ12GP201I Controller
\item Built and demonstrated Zigbee 802.15.4 wireless data transfer software in C to potential business partners
\item Designed, developed, \& tested a random number generation test suite in C\# based on NIST SP800-21
	\begin{itemize}
	\item Performs a bank of statistical confidence interval tests to assure randomness of data for hardware random number generators
	\end{itemize}
\item Developed \& tested C/C\# cryptographic (ECDSA \& SHA-2) software for a software GPS receiver 
\begin{itemize}
\item  Implemented fast galois addition over elliptical curves
\item Tested for cryptorgraphic algorithm validity and security measures.
\end{itemize}
\end{tightitemize}
\sectionspace % Some whitespace after the section
\end{minipage}
%---------RIGHT COLUMN,PAGE 1---------
\newpage% PAGE 2
%---------LEFT COLUMN,PAGE 2---------
\begin{minipage}[t]{0.33\textwidth}
\section{Coursework}

\subsection{Graduate}
Markov Chains\\
Measure Theory\\
Partial Differential Equations (PDEs)\\
Regression Analysis\\
Functional Analysis\\
Point Set Topology\\
Numerical Methods for Interpolation\\
Numerical Methods for PDEs\\
Mathematical Modelling\\
\subsection{Coursera}
Discrete Optimization\\
Linear and Discrete Optimization\\
\subsection{Black Hat 2014}
C and C++ Source Code Auditing\\
Application Security for Developers\\and Attackers\\
\end{minipage}
%---------LEFT COLUMN,PAGE 2---------
%---------RIGHT COLUMN,PAGE 2---------
\begin{minipage}[t]{0.66\textwidth}
\runsubsection{Booz Allen Hamilton Engineering Services, LLC}
\descript{| Technologist}

\location{Continued...}
%%%%%PAGE 2%%%%%
\begin{tightitemize}
\item Designed, developed, \& tested message optimization software for GPS L2C and L5 signals in python
\begin{itemize}
\item Designed a periodic graph which models feasible message sequences
\item Linear inequalities were derived from the L2C and L5 constraints using the periodic graph
\item Message sequencing results we derived using linear programming
\end{itemize}
\item Modeled, developed, \& tested message packing software for GPS C/A signal in python
\begin{itemize}
\item Software was given an a set of messages; using a bin packing problem solver implemented using linear programming, messages were packed into C/A
\end{itemize}
\item Corrected NIST test vectors for SHA-2 based digital signature algorithms
\item Contributed the 'K' in SHAKE for NIST FIPS-202
\item Designed and prototyped a cradle to grave management system for NIST compliant cryptographic keys that met NIST SP800-53 SC-12 (1) and (2)
\item Drafted CONOPS documents for cryptographic systems
\item Drafted command and control software operator manuals
\item Drafted key management plans and non-standard key handling agreements for cryptography systems
\end{tightitemize}
\sectionspace
\section{Research}
\runsubsection{Lie Groups, Homogeneous Manifolds, and Complex Projective Spaces}
\descript{| Co-Authored with Mayra Moran and Atour Bean, May 2009}
Partially funded by NSF Grant DMS-0502258

\runsubsection{Message Optimization Over GPS Civil Navigation Signals}
\descript{| Submitting to ION GNSS+ 2015}
\begin{myframedenv}
In this paper, we pose the problem of maximizing the number of special messages allowed while observing the messaging constraints defined in IS-200 and IS-705 for the GPS signals L2C and L5.  The problem is posed using a graph of feasible message sequencing and then modelling the graphs as linear constraints for a linear programming (LP) problem.
\end{myframedenv}
\sectionspace
\section{Presentations} 
\begin{tightTabularPresentations}
2014     & Permutation and Construction Library: \\
		 &A Library for Permutation Based Cryptography\\
2014	 & What the Heck is Fuzz-Testing?\\
2014	 & BlackHat, DEFCON, SHA-3, \& DIAC: The Summer Conferences\\
2014	 & Configuration Management within Booz Allen Hamilton \\
		 & and an Introduction to C\# ATLAS\\
2014  	 & Message Optimization over L2C and L5\\
2013 	 & Error Correction Codes over Finite Fields \\
2012 	 & Mission Planning Optimization \\
2010 	 & Reed Solomon Error Correction Codes\\
\end{tightTabularPresentations}

\end{minipage}
%---------RIGHT COLUMN,PAGE 2---------
\newpage
%---------LEFT COLUMN,PAGE 3---------
\begin{minipage}[t]{0.33\textwidth}
\hspace{0.33\textwidth}
\end{minipage}
%---------LEFT COLUMN,PAGE 3---------
%---------RIGHT COLUMN,PAGE 3---------
\begin{minipage}[t]{0.66\textwidth}

\section{Miscellaneous Software Development}
\runsubsection{Software Development in Academia}
\location{August 2007 – May 2014 | Cal State Northidge}
\vspace{\topsep}
\begin{tightitemize}
\item Developed a python script to generate random trees using Markov chains
\item Developed R scripts to perform multiregression analysis (ANOVA, $R^2$, principle component analysis)  on car data to model miles per gallon
\item Developed R scripts for bootstapping limited samples to develop statistical tests over the sampling distribution
%%%%%Page 3%%%%%
\item Developed a MATLAB application which was able to identify individuals from their voice using partial differential equations
\item Used C++ OpenCV to identify text in Arabic and English from a digital images.  Application was used for an unmanned air vehicle project
\begin{itemize}
\item Used Canny algorithm, splines, and Hausdorff distance measurements to estimate characters in Arabic and English
\end{itemize}
\item Developed a python application to optimize resource scheduling management software using evolution algorithms
\item Modeled Joukowski air foils (aircraft wing lift) as a system of partial differential equations in Mathematica
\item Modeled growth and decay of animal and bacteria populations as a system of partial differential equations in Mathematica
\end{tightitemize}
\vspace{\topsep}
\runsubsection{Personal Software Development}
\location{June 2004 – Present}
\begin{tightitemize}
\item Currently developing an open source, formally verified, symbolically tested, library for standardized cryptographic permutations and constructions
	\begin{itemize}
	\item Using LLVM KLEE based platforms for symbolic testing
	\item Permutations slated for inclusion are Keccak and any permutation that is selected for the second round of the authenticated encyrption associated data algorithm competition, CAESAR.
	\end{itemize}
\item Developed python code for interpolating stochastic differential equations for use in modern portfolio theory and management
\item Web development in Drupal CMS (version 4,5, and 6)
\end{tightitemize}
\end{minipage}
%---------RIGHT COLUMN,PAGE 2---------
\end{document}